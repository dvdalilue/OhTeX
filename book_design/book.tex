\documentclass[12pt,letterpaper,titlepage,oneside,openright]{book}

\usepackage[utf8]{inputenc}
\usepackage[spanish]{babel}
\usepackage[svgnames]{xcolor}
\usepackage[T1]{fontenc}
\usepackage{amsmath,amsfonts,amsthm}
\usepackage{enumitem}
\usepackage{hyperref}
\usepackage{graphicx}
\usepackage{lettrine}
\usepackage{listings}
\usepackage{relsize}
\usepackage{xspace}
\usepackage{lipsum}
%\usepackage{booktabs}
%\usepackage{cleveref}
%\usepackage{microtype}
%\usepackage{geometry}
%\usepackage{nag}
%\usepackage{siunitx}

\newcommand{\OhTeX}{%
    \makebox[0.76em][c]{O}%
    \makebox[0.25em][c]{%
        \raisebox{0.14em}[0em][0em]{%
            \fontsize{0.5em}{0cm}%
                \selectfont H%
        }%
    }%
    \makebox[1.35em][c]{\TeX}%
}

% Change title of every code listing
\renewcommand*\lstlistingname{Código}
% Code listing options
\lstset{
basicstyle=\footnotesize\ttfamily,
%numbers=left,
%numberstyle=\scriptsize,
%numbersep=1pt,
belowskip=\medskipamount,
frame = single,
framexleftmargin=15pt
}

\renewcommand*\rmdefault{ptm} % Times Font
\renewcommand{\labelitemi}{$\bullet$} % List symbol

% Add a space a the end of the original logo
\newcommand{\latex}{\LaTeX\xspace}
\newcommand{\tex}{\TeX\xspace}
\newcommand{\ohtex}{\OhTeX\xspace}

% Reduce hyphenation
\hyphenpenalty 1000
\exhyphenpenalty 1000

\setlength{\parindent}{3em} % First line paragraph indentation
\setlength{\parskip}{2em} % Paragraph spacing
\renewcommand{\baselinestretch}{1.3} % Line spacing
%1.0    single spacing
%1.3    one-and-a-half spacing
%1.6    double spacing

\setcounter{chapter}{0} % Chapter begins on chapter -1

\newcommand{\nonumberchapter}[1]{
    \chapter*{#1}
    \addcontentsline{toc}{chapter}{#1}
}

\newcommand{\myref}[5]{
    #1. #2. #3. Disponible en Internet: #4, consultado el #5.

% Huerta, Ana. 2003. El gabinete de Historia Natural de Puebla. Disponible en Internet: http://www.elementos.buap.mx/num48/htm/17.htm, consultado el 25 de Abril de 2008.
}

\newcommand{\ttcode}[1]{(\texttt{#1})}

\title{\ohtex \ : Un lenguaje de programación orientado a objetos destinado a la composición tipográfica}
\author{Lilue, David}

\begin{document}

\frontmatter

\maketitle

\nonumberchapter{Preámbulo}

\section*{Resumen}

\lettrine[lines=2]{D}{}\ urante la década de los $70'$ se creó un lenguaje llamado \textit{troff} qué fue inspirado por
runoff, el primer programa para el formato de texto qué fue desarrollado a mediados
de los 60'. \textit{troff} posee la cualidad de de indicar fuente, espaciados,
párrafos, márgenes, pie de páginas, posicionar caracteres de manera arbitraria en las
páginas, superponerlas, entre otras cosas. Aunque hoy en día existen lenguajes para
la composición tipográfica que son comunmente usados, como \LaTeX \ y otros lenguajes de
markup, este sigue siendo usado ampliamente para la documentación de UNIX. Por otro
lado en los años 90' se reimplementado como \textit{groff} por los sistemas GNU, por
cuestiones de software libre. 

Un aspecto importante, es para quién está destinado este lenguaje de composición
tipográfica, es decir, para una comunidad científica y de investigación, para la
comunidad literaria, para una comunidad administrativa y/o gerencial o todas? 

Existen muchos lenguajes de Markup, unos que son simples, otros que son más
complejos, así como \textit{markdown} y \LaTeX, respectivamente. Uno trata de
simplificar y el otro trata de generar un documento enfocado en la belleza. Tomando
en cuenta que se desea la posibilidad de generar documentos similares a los de latex,
pero que sea tan sencillo cómo un lenguaje de markup ligero (LML). Dándole un punto
de vista más estructural y abstracto, haciendo uso de la orientación a objetos.

La idea con este lenguaje es poder manejar de una manera abstracta y simple, los
elementos de un documento así como su composición tipográfica en el documento final
generado. Por lo tanto, será posible usar estructuras de control para generar
distintos documentos a partir de objetos y estructuras abstractas.

% Por ejemplo, agarrar una plantilla con espacios vacíos iterar sobre una lista de
% objetos dónde cada objeto llene de manera diferente y sus espacios vacíos. Sería
% como una meta programación de documentos.

\newpage
 
%\addcontentsline{toc}{chapter}{Dedicatoria}
\section*{} % Dedicatoria
\begin{flushright}
Para todos esos artistas que dejaron su obra, \\ sin su nombre y aun así trascendieron.
\end{flushright}

\newpage

%\addcontentsline{toc}{chapter}{Agradecimientos}
\section*{} % Agradecimientos
\begin{flushleft}
Agradezco a...?
\end{flushleft}

\newpage

\tableofcontents

\mainmatter

%\thispagestyle{empty} % Removes the page numbering.

%\listoffigures

%\pagebreak
 
%\listoftables
 
%\newpage

%\pagenumbering{arabic} % Re-start the page numbering with arabic style.

%\chapter{Motivación}
\chapter{Introducción}
\section{Antecedentes}
%\lipsum[1]
La composición tipográfica es un termino que se maneja desde alrededores del año 1040, refiriéndose
a la estructura del texto ordenando \textit{types} (una pieza que represente una letra o
símbolo) acorde a la ortografía de un lenguaje y al final, obtener la visualización de un texto.

Al principio, se tenían piezas movibles que se ordenaban a mano (dichas piezas estaban al revés 
para que saliesen bien orientadas al imprimir) y se imprimían pagina por pagina. Todo esta 
metodología para la composición tipográfica se inicio en china con piezas de cerámica, 
posteriormente se usaron piezas de metal (bronce). Esta tecnología de impresión llego a Europa
en el siglo 15 y era usada principalmente para cartas.

Para el siglo 19, se había globalizado e industrializado esta forma de impresión, mas que todo por
imprentas y editoriales, gracias a la maquina linotipia (\textit{Linotype Machine}). Estas mismas duraron hasta las ultimas décadas del siglo 20, dado que fueron sustituidas por una nueva tecnología llamada foto-composición, que hacia uso programas para la composición tipográfica electrónica como \textit{troff} y un \textit{phototypesetter} (\textit{CAT}).  A pesar de su innovación, se convirtió rápidamente en un método obsoleto gracias a avanzados lenguajes de composición tipográfica como:

\begin{itemize}
\item ditroff (1979)
\item \TeX \ (1979)
\item \LaTeX \ (1981-1983)
\end{itemize}

\subsection{Linea de Tiempo}

\begin{itemize}

\item 1971, Graphic Systems designs and markets the C/A/T phototypesetter.
\item 1972, Graphic Systems provides a simple computer front end to handle display typesetting from text input using a standard set of fonts.
\item 1973, Bell Laboratories purchases a GSI C/A/T phototypesetter for their UNIX software development team.
\item 1973, The troff software was created by Joe F. Ossanna at Bell Labs.
\item 1974, Graphic Systems partners with Singer Corporation to market the GSI C/A/T phototypesetter. Singer Corporation was allowed to put the Singer C/A/T logo on equipment they sold.
\item 1978, Singer Corporation gets out of the typesetting business.
\item 1978, Wang Laboratories purchases Graphic Systems and continued to market the GSI C/A/T typesetter. for UNIX based phototypesetting. The phototypesetter was then known as the Wang Graphic Systems C/A/T. This allowed the phototypesetter to be available through the 1980s.
\item 1979, ditroff software was created by Brian Kernighan.
\item 1979, TeX software was created by Donald E. Knuth.
\item 1981-1983, LaTeX software was created by Leslie Lamport.
\item 1992, Wang Laboratories declared Chapter 11 bankruptcy.
\end{itemize}

\subsection{\textit{troff}}
\lipsum[1-2]
\subsection{\TeX}
\lipsum[1-2]
\subsection{\LaTeX}
\lipsum[1-2]
\subsection[OhTeX]{\ohtex}
\lipsum[1-2]

\chapter{Marco Teórico}
\lipsum[1-2]
\section{Composición Tipográfica}
\section{Programación orientada a objetos (\textit{OOP})}

\chapter{Marco Tecnológico}
\lipsum[1-2]

\chapter{Desarrollo}
\lipsum[1-2]

\chapter[OhTeX]{\ohtex}

\ohtex es un lenguaje de programación de dominio específico, imperativo,
destinado a la composición tipografica y orientado a objetos. En el diseño del
lenguaje se tienen elementos comumente usados en los lenguajes de programación,
así como estructuras de datos, estructuras de control, funciones,
procediemientos, etc.

\ohtex es un lenguaje de dominio especifico (\textit{DSL}) externo, es decir,
que tiene su propia sintaxis, embedido en Ruby. Esta desición fue tomada porque
Ruby es un lenguaje orientado a objetos y así, todo el trabajo que conlleva el
manejo de objetos en el lenguaje, el alcance dinámico y el modelo de referencia
son trabajo de Ruby. Además, el mismo crea un ambiente propicio para creación de
\textit{DSLs}, haciendolo el mejor candidato para la implementacion de \ohtex.

Ahora, considerando la composición tipografica, se tomo en cuenta a \latex como
herramienta para la generación y estructuración del texto, exportado en
\textit{PDF}, que sería el producto final, es decir, un documento. Por lo tanto,
\ohtex es un traductor (compilador) de Ruby a Latex, demostrando que un
documento puede ser abstraido a un nivel mas alto y mantener una filosofia de
generar documentos bonitos y al mismo tiempo poseer una estructura compresible.

El proceso de traduccion seria asi:

\begin{center}
    \ohtex $\longrightarrow$ Ruby $\longrightarrow$ \latex $\longrightarrow$ PDF
\end{center}

\section[Estructura de un archivo OhTeX]{Estructura de un archivo \ohtex}

Estructura de un programa \ohtex:

\begin{center}
\begin{lstlisting}
    use <paquete>

    <objeto> :: <clase> {
        <atributo> : <valor> 
    }

    <objeto> >> '<nombre_archivo>'
\end{lstlisting}
\end{center}


Durante toda la ejecución del programa, en el archivo pueden crearse $n$
objetos, pero eso no le indica al lenguaje que esos $n$ objetos estarán
presentes en un documento o no, si es que se decide exportar alguno.

\section{Uso de paquetes}

La palabra clave \texttt{use} es usada para indicarle a lenguaje cuales de los
paquetes de latex se deseean usar en cualquiera de los documentos que se vayan a
generar. Algunos de los paquetes exigen uno, mas parametros o son opcionales, en
cualquiera de los casos, la sintaxis para el pase de parametros es por nombre,
como se muestra a continuacion:

\begin{center}
\begin{lstlisting}[caption={Paquetes},label=pkgs]
    use babel { lang : english }
    use microtype { potrusion : true; expasion : true }
    use geometry {
        paper      : letter
        left       : 1in
        right      : 1in
        top        : 1.5in
        headheight : 1in
        headsep    : 0.3in
    }
\end{lstlisting}
\end{center}

Por otro lado, se pueden especificar varios paquetes en una misma linea, eso
solo en el caso de que ninguno de los paquetes necesita parametros, de la
siguiente manera:

\begin{center}
\begin{lstlisting}
    use amsmath amsfonts amsthm
\end{lstlisting}
\end{center}

Esto es, porque en ciertas ocaciones existen paquetes que viene en conjunto ó
tienen un proposito similar, ademas de la flexibilidad en cuanto a no escribir
repetidamente la palabra reservada \texttt{use}.

La herramienta comienza leyendo de arriba a abajo, interpretando, creando
objetos, intereactuando con los mismo y al final puede o no generar uno o mas
documentos. <<\textit{Crear un objeto no implica que el mismo estara en un
documento}>>.

Tomando en cuenta la estructura basica de un programa en \ohtex, en especifico
los \textit{dos-puntos} dobles (\texttt{::}), representa la instanciación de
una clase, es el constructor de objetos y la referencia a esa instancia es dada
por el nombre (identificador) del lado izquierdo (\textit{lhs}). Por ejemplo:

\begin{center}
\begin{lstlisting}
    doc1 :: Article {
        ...
    }   
\end{lstlisting}
\end{center}

Si bien al construir un objeto, el mismo debe inicializarse (en el caso de que
sea necesario), dependiendo si tiene o no valores por defecto. Y para ello, se
usan las llaves, donde se encierra una secuencia (puede ser vacia) de
identificadores que representan algunos (o todos) los atributos del objeto, cada
nombre es seguido del simbolo \textit{dos-puntos} (\texttt{:}) y luego el valor
que inicializará el atributo; cada una de estas inicializaciones es separada por
salto de linea (\verb|\n|) o por punto-y-coma (\texttt{;}). Aqui se muestra una
inicialización (\textit{singleton}):

\begin{center}
\begin{lstlisting}
    font : 10pt
\end{lstlisting}
\end{center}

Siguiendo lo anterior, cual sería el significado semántico de la expresión
anterior? Los candidatos para denotar esta ``asignación" ó ``inicialización"
fueron:

\newpage

\begin{center}
\begin{enumerate}[label=Opción \alph*:, leftmargin=7em]
    \item \hfill \\
          \texttt{=}
    \item \hfill \\
          \texttt{:=}
    \item \hfill \\
          \texttt{:}
    \item \hfill \\
          \texttt{<-}
\end{enumerate}
\end{center}

Y al final, se decidió por \textit{dos-puntos} (\texttt{:}), tomando en cuenta
las siguientes consideraciones:

\begin{itemize}
\item El símbolo se va a escribir seguido, por lo tanto debe poseer la menor cantidad de caracteres posible, es decir, uno.
\item El símbolo debe referir a la asocion de un atributo con un valor, como lo hace CSS (un lenguaje de markup) y JSON (un lenguaje de serializacion de datos).
\item El símbolo debe poseer una semantica acorde a lo que se desea expresar. Por lo tanto, el símbolo (\texttt{:}) representa una pausa para hacer un llamado de atencion en lo que sigue y que siempre esta relacionado con el texto precedente. 
\end{itemize}

El lenguaje manejará tipos enteros, booleanos, clases, instancias de clases,
metodos, funciones y comandos.

A pesar de que el símbolo (\texttt{:}) representa la inicialización de un
atributo de un objeto al momento de su construcción, el simbolo de asignacion es
(\texttt{=}). Ya construido el objeto, se puede acceder y \textit{settear} cada
uno de su atributos. Por ejemplo:

\begin{center}
\begin{lstlisting}
    doc1.font = 11pt
\end{lstlisting}
\end{center}

\section{Lexicografía}

\subsection{Identificadores}

Cada identificador es un nombre que se usa para referir a una varible, metodo,
comando, o paquete. 

De igual manera a como se permite en Ruby y la mayoria de los lenguajes, los
identificadores constan de caracteres alphanumericos (\verb|A-Za-z0-9|) y
underscores (\_), pero sin que comience por numero y que comience siempre por
minúscula. Tambien está permitido, que los identificadores de los metodos puedan
terminar en interrogacion (?), exclamacion (!) ó igual (=). Sin restricciones
en la longitud del identificador. Por ejemplo:

\begin{center}
\begin{lstlisting}
    foo
    bar
    baz
    this_is_my_4er_id
\end{lstlisting}
\end{center}

Por otro lado, existen ciertas palabras reservadas que no podran ser usadas como
identificar en ningun caso. Las cuales se listan mas adelante.

\subsection{Comentarios}

Los comentarios de linea serán a partir de dos puntos seguidos \ttcode{..} hasta
el salto de linea (sin incluirlo). En caso de que desee incluir el salto de
linea en el comentario se deben usar tres puntos seguidos \ttcode{...}. Por
ejemplo:

\begin{center}
\begin{lstlisting}
    .. esta linea no hace nada

    a = 5 .. ``a" es una variable local inicializada con 5

    a = 2 + ... esto es mas util con expresiones largas
        3
\end{lstlisting}
\end{center}

Los comentarios de bloque son a partir de dos puntos y una llave que abre
\ttcode{..\{\}} hasta una llave que cierra seguida de dos puntos \ttcode{\}..}. Por
ejemplo:

\begin{center}
\begin{lstlisting}
    ..{
        Todo lo que se encuentre dentro de este bloque sera
        ignorado. El uso de comentarios de bloques anidados es
        considerado un mala practica pero sucedera de igual
        forma, ..{ asi que es posible hacerlo. }..
    }..    
\end{lstlisting}
\end{center}

\subsection{Palabras reservadas}

La siguientes palabras estan reservadas en \ohtex:

\begin{center}
\begin{lstlisting}
    and
    break
    case
    class
    do
    else
    elsif
    false
    for
    fun
    if
    loop
    nil
    or
    rescue
    return
    self
    super
    true
    try
    until
    use
    when
    while
\end{lstlisting}
\end{center}

\subsection{Expresiones}

Al igual que en Ruby (dado que es un lenguaje embedido), todo es una expresion.
La secuenciacion es dada por los saltos de linea y el punto-y-coma (``;").

\section{Variables y Literales}

\ohtex, al ser un lenguaje embedido en Ruby, mantiene un simil en ciertos
aspectos de su sintaxis, asi como su alcance y declaración de variables. De
alguna manera, es predecible, pero eso ayuda a reducir el tiempo de aprendizaje
y brinda la posiblidad de generar programas intuitivamente, efocandose
escencialmente en la objetivo final, generar un documento con una excelente
composicion tipografica de manera estructural.

\subsection{Variables globales}

Toda variable que comience con dolar \ttcode{\$} tiene alcance global, dicha
variable puede ser accesada en cualquier momento de la ejecucion desde cualquier
lugar. Por ejemplo:

\begin{center}
\begin{lstlisting}
    $foo
\end{lstlisting}
\end{center}

\subsection{Variables locales}

Una variable local es aquella que comienza con un letra en minusculas (a-z) o
underscore (\_). O una llamada a un metodo. Ademas dicha variable sera asequible
en el alcance donde fue declarada, exclusivamente.

\subsection{Variables de instancia}

%Falta definir sintaxis para la definicion de clases

\subsection{Variables de clase}

%Falta definir sintaxis para la definicion de clases

\subsection{Pseudo variables}

Siendo un lenguaje orientado a objetos, debe existir una variable que refiera al
objeto actual, y para ello, tenemos \ttcode{self}.

\section{Formato de texto}

En esta sección se describen varios aspectos para darle un formato especifico al
texto, se podrá ver que existe una similitud con `Markdown` y eso es para
brindar facilidad cuando se genera un programa.

\subsection{Tamaño de Fuente}

\begin{verbatim}
Command             10pt    11pt    12pt
\tiny               5       6       6
\scriptsize         7       8       8
\footnotesize       8       9       10
\small              9       10      10.95
\normalsize         10      10.95   12
\large              12      12      14.4
\Large              14.4    14.4    17.28
\LARGE              17.28   17.28   20.74
\huge               20.74   20.74   24.88
\Huge               24.88   24.88   24.88
\end{verbatim}

\subsection{Cursiva}

Para cambiar el formato de algún texto en especifico, el mismo debe estar entre
dos asteriscos \ttcode{*}, sin espacio al principio y sin saltos de linea.

Siguiendo la siguiente expresión regular en Ruby:

\begin{center}
\verb|/(\*\p{Graph}+\*|\*\\p{Graph}+[\\p{Graph} ]*\*)/|
\end{center}

\subsection{Negrita}

Similar a la sección anterior, se debe agrupar el texto, pero dentro de cuatro
asteriscos `*`, dos de cada lado del texto y siguiendo las mismas restricciones.

\begin{center}
\verb|/(\*\*\p{Graph}+\*\*|\*\*\\p{Graph}+[\\p{Graph} ]*\*\*)/|
\end{center}

\section{Literales}

\subsection{Numeros}

\subsection{Booleanos}

\subsection{Strings}

\subsection{Arreglos}

\subsection{Tablas de Hash}

\subsection{Rangos}

\subsection{Verbatim}

En ciertas ocasiones se desea hacer uso explicito de una expresión que sea
literal en Ruby ó \latex y para ello se hace uso de la siguiente sintaxis:

Ruby:

Cualquier código que sea escrito dentro de las llaves podrá usar los objetos de
archivo definidos en \ohtex u otro verbatim de Ruby.

\begin{verbatim}
&ruby {
    ... código ruby ...
}
\end{verbatim}

\LaTeX:

En el caso de \latex es mas simple, solamente se debe incorporar el código entre
comillas simples al documento, haciendo uso del operador \ttcode{\<\<}. Haciendo
distinción entre el preámbulo y los elementos dentro del documento.

\begin{center}
\begin{lstlisting}
    doc1 :: Article {
        ...
    }

    .. texto ohtex

    doc1.preamble << '
        \newcommand{cmd}[1]{Comando numero #1}
    '

    doc1.document << '
        \noindent
        Example paragraph...
    '
\end{lstlisting}
\end{center}


\chapter{Conclusiones y Recomendaciones}
\lipsum[1-2]
\end{document}