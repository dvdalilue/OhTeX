\documentclass[10pt,letterpaper]{article}

\usepackage[utf8]{inputenc}
\usepackage[spanish]{babel}
\usepackage[T1]{fontenc}
\usepackage{amsmath,amsfonts,amsthm}
\usepackage{xspace}

\setlength{\parskip}{1em}

\newcommand{\OhTeX}{%
    \makebox[0.76em][c]{O}%
    \makebox[0.25em][c]{%
        \raisebox{0.14em}[0em][0em]{%
            \fontsize{0.5em}{0cm}%
                \selectfont H%
        }%
    }%
    \makebox[1.35em][c]{\TeX}%
}

% Add a space a the end of the original logo
\newcommand{\troff}{\textit{troff}\xspace}
\newcommand{\latex}{\LaTeX\xspace}
\newcommand{\tex}{\TeX\xspace}
\newcommand{\ohtex}{\OhTeX\xspace}
\newcommand{\ruby}{\textit{Ruby}\xspace}
\newcommand{\lyx}{\textit{LyX}\xspace}

\begin{document}

La composición tipográfica se ha convertido en una necesidad al momento de
realizar una publicación técnica y científica. Hoy en día existen herramientas que
facilitan la elaboración de documentos de gran complejidad. Durante la década de los
$70'$ se creó el lenguaje \troff, inspirado por \textit{runoff}, el primer programa
para el formato de texto, desarrollado a mediados de los $60'$. \troff posee la
cualidad de darle formato al texto, así como el manejo arbitrario de elementos
tipográficos. Por otro lado, el lenguaje para la composición tipográfica comúnmente
usado, en el ámbito académico, es \latex, pero existen otros lenguajes de
\textit{markup} destinados a otros objetivos.

Un aspecto importante a considerar cuando se elabora un lenguaje de composición
tipográfica es, a quién está destinado el mismo; debe tomarse en cuenta si está
dirigido a una comunidad científica, la comunidad literaria o una comunidad orientada
a la administración y la gerencia. También es posible que se desee una herramienta de
propósito general, que se ajuste a las necesidades de cualquier comunidad. Lenguajes
de \textit{markup}, como \textit{markdown} y \LaTeX, son usados en distintas áreas
para un mismo objetivo, generar un documento; uno trata de simplificar y el otro
trata de generar un documento enfocado en la belleza, respectivamente. En este trabajo se desea la posibilidad de generar documentos similares a los generados por \latex pero que sea sencillo sin una perdida considerable de flexibilidad, dándole un punto de vista más estructural y abstracto, haciendo uso de la orientación a objetos.

\end{document}