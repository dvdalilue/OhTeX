\documentclass[10pt]{beamer}

\usepackage[utf8]{inputenc}
\usepackage[spanish]{babel}
\usepackage[normalem]{ulem}
\usepackage{listings}
\usepackage{xspace}

\newcommand{\bi}{\begin{itemize}}
\newcommand{\ei}{\end{itemize}}
\newcommand{\ig}{\includegraphics}
\newcommand{\subt}[1]{{\footnotesize \color{subtitle} $\gg$ {#1}}}

\renewcommand{\lstlistingname}{Código}

\usetheme{Copenhagen}

\definecolor{mycolor1}{RGB}{88, 116, 152} %blue
\definecolor{mycolor2}{RGB}{88, 112, 88 } %green
\definecolor{mycolor4}{RGB}{232, 104, 80} %red

\definecolor{foreground}{RGB}{26, 26, 26}
\definecolor{background}{RGB}{242,242,242}
%\definecolor{background}{RGB}{24,24,24}
\definecolor{title}{RGB}{255, 216, 0} %yellow
\definecolor{gray}{RGB}{100,100,100}
\definecolor{subtitle}{RGB}{159, 159, 198}
\definecolor{hilight}{RGB}{179, 25, 25}
\definecolor{vhilight}{RGB}{0, 179, 89}

\setbeamercolor*{structure}{fg=mycolor1,bg=white}
\setbeamercolor*{palette primary}{use=structure,fg=title,bg=structure.fg}
\setbeamercolor*{palette quaternary}{fg=background,bg=foreground}

\setbeamercolor*{item projected}{fg=red,bg=mycolor1!50}
\setbeamercolor*{block title example}{fg=white,bg=mycolor1}


\setbeamercolor{titlelike}{fg=title}
\setbeamercolor{subtitle}{fg=subtitle}
\setbeamercolor{institute}{fg=gray}
\setbeamercolor{normal text}{fg=foreground,bg=background}

\setbeamercolor{item}{fg=foreground} % color of bullets
\setbeamercolor{subitem}{fg=gray}
\setbeamercolor{itemize/enumerate subbody}{fg=gray}
\setbeamertemplate{itemize subitem}{{\textendash}}
\setbeamerfont{itemize/enumerate subbody}{size=\footnotesize}
\setbeamerfont{itemize/enumerate subitem}{size=\footnotesize}

\beamertemplatenavigationsymbolsempty % remove footer nav bar
%\setcounter{framenumber}{0}

\makeatother
\setbeamertemplate{footline}
{
    \leavevmode%
    \hbox{%
    \begin{beamercolorbox}[wd=.15\paperwidth,ht=2.25ex,dp=1ex,center]{author in head/foot}%
        \usebeamerfont{author in head/foot}\insertshortauthor
    \end{beamercolorbox}%
    \begin{beamercolorbox}[wd=.70\paperwidth,ht=2.25ex,dp=1ex,center]{title in head/foot}%
        \usebeamerfont{title in head/foot}\insertshorttitle%\hspace*{3em}
        %\insertframenumber{} / \inserttotalframenumber\hspace*{1ex}
    \end{beamercolorbox}%
    \begin{beamercolorbox}[wd=.15\paperwidth,ht=2.25ex,dp=1ex,center]{frame in head/foot}%
        \usebeamerfont{title in head/foot}\insertframenumber{} / \inserttotalframenumber%\hspace*{1ex}
    \end{beamercolorbox}}%
    \vskip0pt%
}
\makeatletter

\title{LENGUAJE DE PROGRAMACIÓN ORIENTADO A OBJETOS PARA LA COMPOSICIÓN TIPOGRÁFICA}
\author{David Lilue}
\institute{Universidad Simón Bolivar}
\date{2016}

\newcommand{\OhTeX}{%
    \makebox[0.76em][c]{O}%
    \makebox[0.25em][c]{%
        \raisebox{0.14em}[0em][0em]{%
            \fontsize{0.5em}{0cm}%
                \selectfont H%
        }%
    }%
    \makebox[1.35em][c]{\TeX}%
}
\newcommand{\ohtex}{\OhTeX\xspace}

\lstset{
basicstyle=\scriptsize\ttfamily,
belowskip=\medskipamount,
frame = shadowbox,
literate={á}{{\'a}}1 {í}{{\'i}}1 {é}{{\'e}}1 {ó}{{\'o}}1 {ú}{{\'u}}1 {ñ}{{\~n}}1,
}

\begin{document}
 
\frame{\titlepage}

\begin{frame}[fragile]
\frametitle{\OhTeX}

\begin{lstlisting}
    use inputenc { encode: utf8 }

    foo :: Article {
        font: 10pt
        paper: letter
    }

    titulo :: Text {
        align: center
        size: LARGE
        text: '\vspace{4em}\OhTeX\vspace{4em}'
    }

    texto :: Text {
        text: '\OhTeX{} es un lenguaje de programación orientado a
        objetos para la composición tipográfica, haciendo uso de
        \LaTeX{} como base para ello. Ofrece la facilidad de manejar
        elementos tipográficos de una manera abstracta y expresiva,
        facilitando la comprensión y aprendizaje del mismo.'
    }

    foo << titulo << texto

    foo >> 'foo'
\end{lstlisting}
\end{frame}

\begin{frame}
\frametitle{\OhTeX}
\begin{center}
\ig[height=0.9\textheight]{images/foo.jpg}
\end{center}
\end{frame}

\begin{frame}[fragile]
\frametitle{\OhTeX}

\begin{lstlisting}
    use inputenc { encode: utf8 }

    foo :: Article {
        font: 10pt
        paper: letter
    }

    titulo :: Text {
        align: center
        size: LARGE
        text: '\vspace{4em}\OhTeX\vspace{4em}'
    }

    texto :: Text {
        text: '\OhTeX{} es un lenguaje de programación orientado a
        objetos para la composición tipográfica, haciendo uso de
        \LaTeX{} como base para ello. Ofrece la facilidad de manejar
        elementos tipográficos de una manera abstracta y expresiva,
        facilitando la comprensión y aprendizaje del mismo.'
    }

    foo << titulo << texto

    foo >> 'foo'
\end{lstlisting}
\end{frame}

\begin{frame}[fragile]
\frametitle{\OhTeX}
\subt{Paquetes}
\begin{lstlisting}
    use inputenc { encode: utf8 }
\end{lstlisting}
\pause
\bi
    \item {\color<3->{gray} Uso de paquetes}
    \pause
    \item {\color<4->{gray} Opciones de paquetes}
    \pause
    \item {\color<5->{gray} Incorporación de inicializacion por nombre}
\ei
\pause
\begin{lstlisting}
    use hyperref, graphicx, listings
\end{lstlisting}
\pause
\vfill
\rule{0.5\textwidth}{0.3pt}\\
{\scriptsize \color{hilight} Similar a como se hace en \LaTeX}
\end{frame}

\begin{frame}[fragile]
\frametitle{\OhTeX}
\subt{Contrucción de instancia}
\begin{center}
{\small
\textbf{Instancias de documentos}}
\end{center}
\begin{lstlisting}
    foo :: Article {
        font: 10pt
        paper: letter
    }
\end{lstlisting}
\pause
\bi
    \item {\color<3->{gray} Clases de documentos}
    \pause
    \item {\color<4->{gray} Simbolos `\texttt{::}' y `\texttt{:}'}
    \pause
    \item {\color<5->{gray} Atributo-Valor}
    \pause
    \item {\color<6->{gray} Unidades y variables definidas}
\ei

\pause
\vfill
\rule{0.5\textwidth}{0.3pt}\\
{\scriptsize \color{hilight} El uso del simbolo `\texttt{:}' fue influenciado por CSS}
\end{frame}

\begin{frame}[fragile]
\frametitle{\OhTeX}
\subt{Formato de texto}
\begin{center}
{\small
\textbf{Instancias de texto}}
\end{center}
\begin{lstlisting}
    titulo :: Text {
        align: center
        size: LARGE
        text: '\vspace{4em}\OhTeX\vspace{4em}'
    }

    texto :: Text {
        text: '\OhTeX{} es un lenguaje de programación orientado a
        objetos para la composición tipográfica, haciendo uso de
        \LaTeX{} como base para ello. Ofrece la facilidad de manejar
        elementos tipográficos de una manera abstracta y expresiva,
        facilitando la comprensión y aprendizaje del mismo.'
    }
\end{lstlisting}

\bi
    \pause
    \item {\color<3->{gray} Formato de texto como un atributo}
    \pause
    \item {\color<4->{gray} Usa de \textit{verbatim} \LaTeX{} en \OhTeX{}}
\ei

\pause
\vfill
\rule{0.5\textwidth}{0.3pt}\\
{\scriptsize \color{hilight} Integración, manteniendo similitudes}

\end{frame}

\begin{frame}[fragile]
\frametitle{\OhTeX}
\subt{Incorporación de elementos tipográficos y salida}

\begin{lstlisting}
    foo << titulo << texto

    foo >> 'foo'
\end{lstlisting}

\pause
\bi
    \item {\color<3->{gray} Operador `\texttt{<<}' y `\texttt{>>}'}
    \pause
    \item {\color<4->{gray} Estructuración de un documento}
    \pause
    \item {\color<5->{gray} Composición referencial}
\ei

\pause
\vfill
\rule{0.5\textwidth}{0.3pt}\\
{\scriptsize \color{hilight} Crear un elemento tipográfico no lo incluye en un documento}
\end{frame}

\begin{frame}[fragile]
\frametitle{\OhTeX}
\subt{Proceso de traducción}
\begin{columns}[c] % the "c" option specifies center vertical alignment
    \column{.15\textwidth} % column designated by a command
        \begin{enumerate}
            \item \OhTeX
            \item \textit{Ruby}
            \item \LaTeX
            \item PDF
        \end{enumerate}
    \pause
    \column{.85\textwidth}
    \begin{center}
    {\bf \LaTeX{} generado}
    \end{center}
    \begin{lstlisting}
    \documentclass[10pt,letterpaper]{article}
    \usepackage[utf8]{inputenc}
        
    \begin{document}
    \begin{center}
    {\LARGE \vspace{4em}\OhTeX\vspace{4em}}
    \end{center}

    {\normalsize \OhTeX{} es un lenguaje de programación
        orientado a objetos para la composición
        tipográfica, haciendo uso de \LaTeX{} como base
        para ello. Ofrece la facilidad de manejar elementos
        tipográficos de una manera abstracta y expresiva,
        facilitando la comprensión y aprendizaje del mismo.
    }
    \end{document}
    \end{lstlisting}
\end{columns}

\pause
\vfill
\rule{0.5\textwidth}{0.3pt}\\
{\scriptsize \color{hilight} Puede que sean ``menos'' caracteres. Eso no significa mayor expresividad}
\end{frame}

\begin{frame}[fragile]
\frametitle{\OhTeX}

\begin{lstlisting}
    use inputenc { encode: utf8 }

    foo :: Article {
        font: 10pt
        paper: letter
    }

    titulo :: Text {
        align: center
        size: LARGE
        text: '\vspace{4em}\OhTeX\vspace{4em}'
    }

    texto :: Text {
        text: '\OhTeX{} es un lenguaje de programación orientado a
        objetos para la composición tipográfica, haciendo uso de
        \LaTeX{} como base para ello. Ofrece la facilidad de manejar
        elementos tipográficos de una manera abstracta y expresiva,
        facilitando la comprensión y aprendizaje del mismo.'
    }

    foo << titulo << texto

    foo >> 'foo'
\end{lstlisting}
\end{frame}

%\begin{frame}[fragile]
%\frametitle{\OhTeX}
%
%\end{frame}
%
%\begin{frame}[fragile]
%\frametitle{\OhTeX}
%
%\end{frame}

\begin{frame}{Contenido}
\bi
    \item Origen
    \item Influencias y Allegados
    \item Implementación
    \item Manejo de objetos tipográficos
    \item Estructuración de un documento
    \item Conclusión
\ei
\end{frame}

\begin{frame}{Origen}
\bi
    \item Bi Sheng. China. Mediados del siglo 11
        \bi
            \item \textit{Types} movibles de cerámica
        \ei
    \pause
    \item Corea. Siglo 13-14
        \bi
            \item \textit{Types} movibles metálicos
        \ei
    \pause
    \item Johannes Gutenberg. Magnuncia, Alemania. Mediados del siglo 15
        \bi
            \item Prensa
        \ei
    \pause
    \item Ottmar Mergenthaler. Finales del siglo 19
        \bi
            \item Maquina Linotapia
        \ei
    \pause
    \item Foto-composición tipográfica. 1970-1980.
    \pause
    \item Composición tipográfica digital.
        \bi
            \item \TeX. Donald Knuth. 1978
            \item \LaTeX. Leslie Lamport. 1985
        \ei
\ei

\pause
\vfill
\rule{0.5\textwidth}{0.3pt}\\
{\scriptsize \color{hilight} El libro más viejo impreso con \textit{types} movibles es el \textit{Jikji}}
\end{frame}

\begin{frame}{Origen}
\begin{columns}[c]
    \column{.5\textwidth}
        \begin{center}
            \ig[height=\textwidth]{images/type.png}
        \end{center}
    \column{.5\textwidth}
        \begin{center}
            \ig[height=0.8\textwidth]{images/movable_type.jpg}
        \end{center}
\end{columns}
\end{frame}

\begin{frame}{Contenido}
\bi
    \item \sout{Origen}
    \item Influencias y Allegados
    \item Implementación
    \item Manejo de objetos tipográficos
    \item Estructuración de un documento
    \item Conclusión
\ei
\end{frame}

\begin{frame}{Influencias y Allegados}
\bi
    \pause
    \item Herramientas de composición tipográfica
        \bi
            \item Lenguajes de marcado
            \item \TeX/\LaTeX
        \ei
    \pause
    \item Lenguaje de programación orientados a objetos
        \bi
            \item Ruby
        \ei
    \pause
    \item Lenguaje de estilo y formato
        \bi
            \item CSS
        \ei
    \pause
    \item Notación legible por humano
        \bi
            \item JSON
        \ei
    \pause
    \item Hermano?
        \bi
            \item Curl
        \ei
\ei
\end{frame}

\begin{frame}{Influencias y Allegados}
\bi
    \pause
    \item Aplicaciones integradas con \LaTeX
        \bi
            \item LyX, TeXmacs (no \TeX)
            \item Texmaker
            \item Overleaf, ShareLaTeX
        \ei
    \pause
    \item \OhTeX. Lenguaje de dominio específico
        \bi
            \item Que tan específico?
            \item Ventajas y Desventajas
        \ei
    \pause
    \item \OhTeX. Lenguaje embebido
        \bi
            \item Ruby
            \item \LaTeX
            \item Tiene que ser embebido?
        \ei
\ei

\pause
\vfill
\rule{0.5\textwidth}{0.3pt}\\
{\scriptsize \color{hilight} WYSIWYM: \textit{what-you-see-is-you-mean}. WYSIWYG: \textit{what-you-see-is-you-get}}
\end{frame}

\begin{frame}{Contenido}
\bi
    \item \sout{Origen}
    \item \sout{Influencias y Allegados}
    \item Implementación
    \item Manejo de objetos tipográficos
    \item Estructuración de un documento
    \item Conclusión
\ei
\end{frame}

\begin{frame}{Implementación}
\bi
    \onslide<1->{
    \item \textit{Lexer} y \textit{Parser}
        \bi
            \item Por qué Ruby?
        \ei
    }
    \onslide<2->{
    \item De librería a DSL
    }
    \onslide<3->{
    \item Consecuencias de usar dos intermediarios
        \bi
            \item Cuales son esenciales?
            \item Por qué dejar archivos intermedios?
        \ei
    }
    \onslide<4->{
    \item Consideración de opciones
    }
\ei

\vfill
\onslide<2->{
\rule{0.5\textwidth}{0.3pt}\\
{\scriptsize \color{hilight} DSL: \textit{Domain specific language}}
}
\end{frame}

\begin{frame}{Contenido}
\bi
    \item \sout{Origen}
    \item \sout{Influencias y Allegados}
    \item \sout{Implementación}
    \item Manejo de objetos tipográficos
    \item Estructuración de un documento
    \item Conclusión
\ei
\end{frame}

\begin{frame}[fragile]
\frametitle{Manejo de objetos tipográficos}
\subt{Estructuras de control}

\begin{lstlisting}
    tabla1 :: Table {
        header : ['Número', 'Cuadrado']
    }

    valores = [1,2,3,4,5,6,7,8,9]

    for i in values {
        tabla1.add_row([i, i*i])
    }
\end{lstlisting}
\end{frame}

\begin{frame}[fragile]
\frametitle{Manejo de objetos tipográficos}
\subt{Estructuras de control}

\begin{lstlisting}
    nombre :: Text {
        size  : Large
        align : left
    }

    .. Suponiendo que tenemos un objeto membrete y un texto 
    foo << membrete << nombre << lipsum1

    nombres = [
        'Sr. Doe',
        'Sra. Jane',
        'Dr. John',
        'Prof. Jenny'
    ]

    for r in nombres {
        nombre.text = r
        foo >> r
    }
\end{lstlisting}

\pause
\vfill
\rule{0.5\textwidth}{0.3pt}\\
{\scriptsize \color{hilight} Generar documentos similares a través de estructuras de control}
\end{frame}

\begin{frame}[fragile]
\frametitle{Manejo de objetos tipográficos}
\subt{Referencia a un mismo objeto}

\begin{lstlisting}
    ..{ Suponiendo que tenemos un objeto 'encabezado' usado
        comunmente en documentos administrativos y/o repetitivos }.. 
    
    carta :: Letter
    acta :: Article
    memo :: Memo

    carta << encabezado << nombre1 << texto1

    acta << encabezado << contenido

    memo << encabezado << titulo << texto2

    ...
\end{lstlisting}

\pause
\vfill
\rule{0.5\textwidth}{0.3pt}\\
{\scriptsize \color{hilight} El uso de un identificador para un objeto tipográfico evita reescribirlo}
\end{frame}

\begin{frame}{Contenido}
\bi
    \item \sout{Origen}
    \item \sout{Influencias y Allegados}
    \item \sout{Implementación}
    \item \sout{Manejo de objetos tipográficos}
    \item Estructuración de un documento
    \item Conclusión
\ei
\end{frame}

\begin{frame}[fragile]
\frametitle{Estructuración de un documento}
\bi
    \item \LaTeX. Secuencia de elementos tipográficos
    %\item \OhTeX. Árbol de objetos tipográficos
\ei

\begin{lstlisting}
    \documentclass[10pt,letterpaper]{book}
    \usepackage[spanish]{babel}

    \begin{document}
    % elementos tipográficos
    
    \section{Sección 1}

    \begin{itemize}
        % Items
    \end{itemize}

    \section{Sección 2}

    % texto lipsum

    % elementos tipográficos
    \end{document}
\end{lstlisting}
\end{frame}

\begin{frame}[fragile]
\frametitle{Estructuración de un documento}
\bi
    %\item \LaTeX. Secuencia de elementos tipográficos
    \item \OhTeX. Árbol de objetos tipográficos
\ei

\begin{lstlisting}
    main :: Book {
        font : 10pt
        paper : letter
    }

    seccion1 :: Section { name : 'Sección 1' }
    seccion2 :: Section { name : 'Sección 2' }

    lista :: List {
        items : [...]
    }

    seccion1 << lista

    texto :: Text {
        text : .. lipsum
    }

    seccion2 << texto

    main << seccion1 << seccion2
\end{lstlisting}
\end{frame}

\begin{frame}[fragile]
\frametitle{Estructuración de un documento}
\bi
    %\item \LaTeX. Secuencia de elementos tipográficos
    \item \OhTeX. Árbol de objetos tipográficos
\ei
\begin{center}
\ig[height=0.5\textheight]{images/arbol.png}
\end{center}

\pause
\vfill
\rule{0.5\textwidth}{0.3pt}\\
{\scriptsize \color{hilight} La estructuración se asemeja a un tabla de contenido}
\end{frame}

\begin{frame}{Contenido}
\bi
    \item \sout{Origen}
    \item \sout{Influencias y Allegados}
    \item \sout{Implementación}
    \item \sout{Manejo de objetos tipográficos}
    \item \sout{Estructuración de un documento}
    \item Conclusión
\ei
\end{frame}

\begin{frame}{Conclusión}

\bi
    \item Expresividad
    \item Dificultad de aprendizaje
    \item Usabilidad
    \item Lenguajes anfitriones
    \item Implementación
    \item Diversos gustos personales
\ei

\end{frame}

% \begin{frame}{It's all about money}
% \bi
%     \item {\color<3| handout 0>{hilight} Research}
%     \item {\color<3| handout 0>{hilight} Writing}
%     \item {\color<3| handout 0>{hilight} Peer review, editorial oversight}
%     \item {\color<4| handout 0>{hilight} Journal administration}
%     \item {\color<4| handout 0>{hilight} Copy editing, typesetting}
%     \item {\color<4| handout 0>{hilight} Distribution}
%     \item<2->{\color<2| handout 0>{vhilight} \color<4| handout 0>{hilight} Profit}
% \ei
% \end{frame}
\end{document}  